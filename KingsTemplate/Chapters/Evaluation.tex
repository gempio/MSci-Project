\chapter{Evaluation}
    \section{Project Progress Evaluation at the time of report}
      This section will largely focus on evaluating the progress made during the report. It will discuss the original goals against what was achieved and what was missing in the project. What things could have been done better? What things have been done well? Why have some things not worked out quite the way they should(Think robots overlapping each other or not constantly upgrading the map or robots not proposing what rooms to go to). Also discuss the help that was reached out for and critique the ROS community as they are difficult at times. Discuss where the project could be applied? Also make a note how compared to real life this map was really small and usually robots will have much bigger environments to transverse through. This means that if time is of essence, supporting multiple robots is absolutely critical and being scalable is another issue to look at which the project did focus on in code design.

    \section{How the project compares with other research}
      Following that, discuss how this project compares to other papers previously stated in the background and as such. In what way was this project something completely different(Think humans not in the same environment as robots). How was it similar?(Think about the collaboration papers focusing on fluency). For more ask at tomorrows meeting what can be discussed here. I could also discuss the inverse property of the project against projects like Amazon Drone Delivery.

    \section{Future possibilities for expanding the project}
      Should I write something here?

      \subsubsection{Minimizing dependencies}
        Complete care was given in the project to made the code as scalable as possible through the use of non discriminating functions(Functions that apply to all robots) through Object Oriented Programming etc. But some functions in ROS do require the user to create separate file for each robot and as such, more elaborate ways could be thought of to allow a more Object Oriented approach to be taken. With that, the scalability of the project could increase to the point that it would turn more into a non-deterministic framework that allows plug-ins to be taken in.

      \subsubsection{Automatic area clustering of robots}
        Based on research done in Sensor networks, energy limitations are a huge concern there and multitude of research was carried out on maximizing the lifetime of networks. As such, motivation can be given to displacing the robots throughout the map to maximize their lifetime and guide the user better around the game. This could be done through the use of clustering algorithms for robots to negotiate on which part of the map they will focus on. In this project this will take on a bigger challenge as in some scenarios where environments are hostile robots could be lost along their path and as such two robots could agree to search regions of the map.

      \subsubsection{Start using physical robots for the Game}
        The project stopped at using simulations to run the game. In this way it is not much different than using simple game software to simulate such an environment. Creating a framework using ROS actually allows the project to transfer from simulation to real life is actually fairly straightforward with the foundations already built up. With the robot controller, all functions are sent to specific robots odometry and since odometry will always be present the next focuses in the project could be to start implementing real life robots with maps uploaded onto them.

        In such scenarios a lot of issues would be overcome with robots creating real time data and submitting them to the map as well as avoiding obstacles that could be other robots. In such a case study users would have access to robots but not see the maze themselves. It would add excitement to the game.

      \subsubsection{Expand the GUI to JavaFX}
        Currently, the GUI is built using Java Swing which in itself is a very formal language that is usually used to build business like components with little support for drawing extra objects or on panes. There are packages such as AWT but their use is widely discouraged by the community and could stall the project more.

        Instead a proposition of using JavaFX is in place. This package is slowly used more frequently to add flexibility to otherwise difficult to handle Java GUI design. In the end the projects audience is what will determine the projects quality so ensuring a solid GUI could be the way to move forward.

      \subsubsection{Add Phone support}
        While a minor feature requiring to only boot up another GUI, gaming on the phone is becoming more and more of a standard. With that in mind giving users the freedom to not carry around a laptop to events where the game might be played, using hotspot routers for the game and allowing applications to play the game on top of computers could be in fact a very useful feature. There are numerous possibilities for implementing such a feature but there are also technologies that allow writing applications that export to all phone platforms. Discussing them however falls outside of the scope of the project.

      \subsubsection{Flexibility through Multi ROS Support}
        If the environment stayed as it is, creating VM's that run multiple ROS environments and handle them accordingly could be an approach to create a gaming environment that multiple users could enjoy online whenever they would choose to wish so. Such environment would work by using name assigning and would require adding another Server node that would handle multiple instances of the current server class that would populate itself over the machines ports with sockets. 

        This would create a scalable game environment that potentially an unlimited number of users could use. This could open opportunities for competitions in the game where based on a scoring system of choice users could compete in robotics challenges that use actual real robot operating system as opposed to simulations that wouldn't otherwise give them an experience of using a \"real\" robot in its environment.
      
      \subsubsection{Deep sea exploration}
        ROS currently lacks the support for 3D path planning. This is not to say that 3D control, that is actually available but in the future, plans might extend to ros being able to support 3D path planning and then another world of opportunity will open up in Deep sea exploration. The current project allows for communication between ROS and whatever GUI is presented so its not difficult to see how inputting the right packages for such implementations could be extremely useful. Already a lot of research has been going into node deployments in deep sea and how to use these to tackle communication however, map building of deep sea ocean and understanding more about what's beneath us is an area of research that could benefit from the framework suggested in this project where robots are given coordinates through a server allowing for a customizable GUI on the side of developers for such projects.

      \subsubsection{Search-And-Rescue Scouting}
        This is actually an extension to the paper by Govindarajan\cite{Vijay} where robots would assist a person to minimize time spent in a single room during search and rescue missions. This project could be an extension to that but much rather use robots for pre-made environments that suffered some kind of trauma to weight risks of assisting certain people and search for survivors in places that have the best chance of making it out. Building maps of risky environments could give rescuers chances of creating robust plans that would maximize the number of people saved in such harsh and difficult times.