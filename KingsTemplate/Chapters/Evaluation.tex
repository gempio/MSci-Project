\chapter{Evaluation}
  The project ended with a successful implementation of the system with certain compromises that had to be made throughout. With the product at the current stage it was possible to use the dialogue framework with the users to produce complete runs and results where users would trust robots about visiting rooms than themselves. This opens opportunities in research where such behaviors are examined like technology impact on human life. As such the system can be considered successful for user testing.

  The system also incorporates the dialogue framework which allows argumentation based research to occur in human-robot teams meaning that it is a viable tool for research with this domain of computer science.

  The evaluation here will compare what was the original intention behind the system as compared to what has been achieved in the project. Further sections will then analyze what could be improved on based on what wasn't achieved and possibilities for improvements to the overall project outside of the specification.

  \section{What was achieved vs. original intentions}
    The project started off with trying to achieve human robot collaboration with support for multiple robots and possibly extending the project to allow robots to take some of the control from the user. Majority of this system was achieved with the multi robot collaboration having to be taken out. This was mostly due to the time constraints for such a complex project. There were also some features that the system did not develop be it because of time or because of the software limitations. 

    The fully developed product boasts features that range from component integration over various platforms of programming languages to the simplest features like providing the user with an interface and a dialogue framework to interact with and although working with ROS delayed the project by substantial amount for getting it working properly, ROS has proven a great tool once it was up and running providing additional functionality that would normally take much longer to implement.

    Considering the project meets most of the desired specification criteria, things that were not finished are:
    \begin{itemize}
      \item Real time map update for the robot positions.
      \item Robots being aware of each others presence.
      \item Various sentence models for the same type of question in the dialogue.
      \item Keyboard shortcuts for the GUI
    \end{itemize}

    Three of the points on the list however are convenience based and were mostly limited by time constraints on the project. In the future these could potentially be the first to be implemented if there was a chance to do so.

    The one vital point is that in the simulation environment there was currently no way(or none found yet) to update the robot maps with the other robots positions due to the glitch with layered costmaps in ROS. This problem however is only with regards to the static maps and the user will still be able to take full advantage of the ROS stack to navigate through the system if the system was brought to the physical world. This is promising as it shows that while ROS may be lacking in simulation environments, the possibilities for using the created framework in physical environments will show a multitude of uses.

  \section{How the project compares with other research}
    Following that, discuss how this project compares to other papers previously stated in the background and as such. In what way was this project something completely different(Think humans not in the same environment as robots). How was it similar?(Think about the collaboration papers focusing on fluency). For more ask at tomorrows meeting what can be discussed here. I could also discuss the inverse property of the project against projects like Amazon Drone Delivery.

  \section{Future possibilities for expanding the project}
    The system is versatile enough that there are numerous ways in which it can expand two of which can be broken down into improvements on the current system and specialization of the system into different areas. These two variants are listed in the following subsections to outline various possibilities that the system could take:

    \subsection{Improvements}
    The system has been fully developed and is capable of providing functionality for the argumentation-style dialogue research for multi-robot teams however it is important to note that this functionality can be improved on to blend the barrier between the use of the framework and the user. This will ensure that the users can connect with the environment better and provide better feedback and how the dialogue affected their results rather than the UI bothering them because it is not interactive enough. There could of course be more modifications to the code. As such following improvements are mentioned as part of possible improvements.

    \subsubsection{Minimizing dependencies}
      Complete care was given in the project to made the code as scalable as possible through the use of non discriminating functions(Functions that apply to all robots) through Object Oriented Programming etc. But some functions in ROS do require the user to create separate file for each robot and as such, more elaborate ways could be thought of to allow a more Object Oriented approach to be taken. With that, the scalability of the project could increase to the point that it would turn more into a non-deterministic framework that allows plug-ins to be taken in.

    \subsubsection{Automatic area clustering of robots}
      Based on research done in Sensor networks, energy limitations are a huge concern there and multitude of research was carried out on maximizing the lifetime of networks. As such, motivation can be given to displacing the robots throughout the map to maximize their lifetime and guide the user better around the game. This could be done through the use of clustering algorithms for robots to negotiate on which part of the map they will focus on. In this project this will take on a bigger challenge as in some scenarios where environments are hostile robots could be lost along their path and as such two robots could agree to search regions of the map.

    \subsubsection{Start using physical robots for the Game}
      The project stopped at using simulations to run the game. In this way it is not much different than using simple game software to simulate such an environment. Creating a framework using ROS actually allows the project to transfer from simulation to real life is actually fairly straightforward with the foundations already built up. With the robot controller, all functions are sent to specific robots odometry and since odometry will always be present the next focuses in the project could be to start implementing real life robots with maps uploaded onto them.

      In such scenarios a lot of issues would be overcome with robots creating real time data and submitting them to the map as well as avoiding obstacles that could be other robots. In such a case study users would have access to robots but not see the maze themselves. It would add excitement to the game.

    \subsubsection{Expand the GUI to JavaFX}
      Currently, the GUI is built using Java Swing which in itself is a very formal language that is usually used to build business like components with little support for drawing extra objects or on panes. There are packages such as AWT but their use is widely discouraged by the community and could stall the project more.

      Instead a proposition of using JavaFX is in place. This package is slowly used more frequently to add flexibility to otherwise difficult to handle Java GUI design. In the end the projects audience is what will determine the projects quality so ensuring a solid GUI could be the way to move forward.

    \subsubsection{Mobile Application support}
      While a minor feature requiring to only develop another GUI, gaming on the phone is becoming more and more of a standard. With that in mind giving users the freedom to not carry around a laptop to events where the game might be played, using hotspot routers for the game and allowing applications to play the game on top of computers could be in fact a very useful feature. There are numerous possibilities for implementing such a feature but there are also technologies that allow writing applications that export to all phone platforms. Discussing them however falls outside of the scope of the project.

    \subsubsection{Flexibility through Multi ROS Support}
      If the environment stayed as it is, creating VM's that run multiple ROS environments and handle them accordingly could be an approach to create a gaming environment that multiple users could enjoy online whenever they would choose to wish so. Such environment would work by using name assigning and would require adding another Server node that would handle multiple instances of the current server class that would populate itself over the machines ports with sockets. 

      This would create a scalable game environment that potentially an unlimited number of users could use. This could open opportunities for competitions in the game where based on a scoring system of choice users could compete in robotics challenges that use actual real robot operating system as opposed to simulations that wouldn't otherwise give them an experience of using a \"real\" robot in its environment.

    \subsection{Migration from dialogue research}
      In the future, the software stack incorporated into the project does not have to follow its intended destination. The project has been developed with customizability in mind and as such bits and pieces can be taken in and out to create an environment that suits other needs. There are numerous exciting possibilities to port the project to however the two that are taken into consideration to maximize the use of the current framework are:

      \subsubsection{Deep sea exploration}
        ROS currently lacks the support for 3D path planning. This is not to say that 3D control, that is actually available but in the future, plans might extend to ros being able to support 3D path planning and then another world of opportunity will open up in Deep sea exploration. The current project allows for communication between ROS and whatever GUI is presented so its not difficult to see how inputting the right packages for such implementations could be extremely useful. Already a lot of research has been going into node deployments in deep sea and how to use these to tackle communication however, map building of deep sea ocean and understanding more about what's beneath us is an area of research that could benefit from the framework suggested in this project where robots are given coordinates through a server allowing for a customizable GUI on the side of developers for such projects.

      \subsubsection{Search-And-Rescue Scouting}
        This is actually an extension to the paper by Govindarajan\cite{Vijay} where robots would assist a person to minimize time spent in a single room during search and rescue missions. This project could be an extension to that but much rather use robots for pre-made environments that suffered some kind of trauma to weight risks of assisting certain people and search for survivors in places that have the best chance of making it out. Building maps of risky environments could give rescuers chances of creating robust plans that would maximize the number of people saved in such harsh and difficult times.